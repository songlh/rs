\documentclass[10pt]{article}
\usepackage{geometry}
\usepackage{amsmath}

\usepackage{url}
\addtolength{\oddsidemargin}{-.4in}
\addtolength{\evensidemargin}{-.4in}
\addtolength{\textwidth}{0.8in}

\addtolength{\topmargin}{-0.3in}
\addtolength{\textheight}{1.3in}
\usepackage[numbers, sort]{natbib}

\usepackage{paralist}
\newenvironment{ParaEnum}[0]{\begin{inparaenum}[(1)]}{\end{inparaenum}}

\title{\vspace{-.7in}\bf{Linhai Song - Research Statement\vspace{-.4in}}}
%\author{Guoliang Jin}
\date{}
\begin{document}
\maketitle\vspace{-.2in}

My research interests span the areas of systems, security, and software engineering.
The goal of my research is to help developers deliver more efficient, reliable, and secure software systems
from the perspective of combating various types of software bugs.  
In particular, my PhD research focuses on performance bugs, while my post-PhD
research centers around concurrency and safety issues in new programming languages (e.g., Go, Rust).
My experience spans different stages of fighting software bugs:
real-world bug understanding~\cite{jin12perfbug,go-asplos,yu2019fearless,sosp-boqin}, 
bug detection during testing~\cite{jin12perfbug,Nistor13ICSE,icse-boqin}, 
bug diagnosis for user-reported failures~\cite{Song14OOPSLA,Song17ICSE}, 
and automated bug fixing~\cite{jin11afix,Song14MICRO}. 
My expertise covers the whole stack of systems, 
from hardware to operating systems and applications. 

Two characteristics describe my research. 
First, it is driven by a deep understanding of real-world problems. 
Before diving into a solution, 
I always validate the importance and relavence of 
the problem I am going to address. 
As a result, I led several empirical studies to understand real-world bugs and 
malware, which offer benchmarks to evaluate my built techniques 
and also reveal new research problems I did not perceive.
Second, my research is interdisciplinary.
I applied classic algorithms in programming languages, 
new hardware and machine learning models to solve various problems 
in operating systems, software engineering and security. 
As a result, I have built up a solid understanding of both 
techniques and practical problems, and my research
has appeared in a diverse set of top-tier 
venues in programming languages, 
operating systems and software engineering including 
PLDI, ASPLOS, OOSPLS, ICSE, FSE and MICRO.

My research work has already gained industrial and academic impact. 
My two performance bug detection techniques~\cite{jin12perfbug, Nistor13ICSE} 
found hundreds of previously unknown performance bugs in mature open-source software, 
many of which have already been confirmed and fixed by developers. 
My performance bug fixing project~\cite{Song14MICRO} was 
runner-up in the competition 
for best paper at MICRO-47 (2014)
and has also led to an issued patent~\cite{comppatent}.
My concurrency bug fixing system~\cite{jin11afix} won 
the ACM SIGPLAN Research Highlights Award~\cite{afixnom}. 
As of now, my publications have 717 citations in total. 
My research also gets media coverage 
from {\it the Morning Paper}, {\it Jexia}, 
{\it I Programmer} and {\it InfoQ}.
In the following, I will describe previous research experience 
and future research directions. 

\vspace{-.1in}
\section{My PhD Work: Combating Real-World Performance Bugs}

My PhD work is mainly focused on addressing the software 
inefficiency problem from the perspective of combating performance bugs. 
Performance bugs are software defects that can cause inefficient execution. 
My methodology is to investigate existing approaches originally designed for functional bugs
and try to apply, adapt, and extend them for performance bugs. 


\vspace{-.1in}
\paragraph{Real-world Performance Bug Understanding.}
Research on performance bugs, like that on functional bugs, 
should be guided by empirical studies. 
In order to improve the understanding of real-world performance bugs, 
I co-led what we believe to be the first 
empirical study of real-world performance bugs, 
based on 110 bugs randomly sampled from five open-source software suites~\cite{jin12perfbug}. 
Our study examined four dimensions 
that cover the complete lifetime of performance bugs:
their root causes, 
how they are introduced, their triggering conditions, and their fixing strategies. 
Our study provides many interesting and important findings, 
such as many performance bugs are introduced by workload mismatch 
and misunderstanding of APIs' performance features; 
and around half of the studied performance bugs 
will only manifest under inputs with both special features and large scales. 
Our study can guide future research on performance bugs.
According to Google Scholar, our study has already been cited 
242 times since 2012. 
It has already motivated our own performance bug detection 
and performance failure diagnosis projects.



\vspace{-.1in}
\paragraph{Static/Dynamic Performance Bug Detection.}
Bug detection aims to find previous unknown bugs,
before they manifest in front of end users. 
Rule-based detection techniques have been widely used to identify 
functional bugs in previous literatures.
Our study shows that both statically checkable efficiency rules 
and violations of these rules exist widely in software. 
Inspired by this finding,
we built a suite of rule-based static checkers for performance problems~\cite{jin12perfbug}.
In total, our checkers found 332 previously unknown performance bugs 
from the latest versions of Apache, Mozilla, and MySQL. 
We reported some of the identified bugs to developers. 
Of these, 77 have been confirmed by developers so far, 
and 15 reported bugs have already been fixed.
Our empirical study also finds that 90\% of performance bugs involve loops, 
and 50\% of performance bugs involve at least two levels of loops. 
Motivated by this finding, I helped a fellow graduate student 
build a novel automated test oracle named Toddler~\cite{Nistor13ICSE},
which can leverage the well-established testing process for functional bugs to 
identify performance bugs caused by inefficient nested loops. 
Using Toddler, we found 42 new bugs in six Java projects.
Based on our bug reports, developers so far have fixed 21 of these bugs 
and confirmed 6 more as real bugs. 

\vspace{-.1in}
\paragraph{Performance Failure Diagnosis.}
As another key aspect of fighting performance problems,
diagnosing a user-perceived performance failure is to figure out which code 
region is responsible for the observed slowdown and 
to design a patch to optimize the code region.
To improve performance failure diagnosis, 
we first explored how to 
apply statistical debugging to performance bugs~\cite{Song14OOPSLA}. 
Our empirical study shows that performance failures are usually observed through comparison,
and when reporting a performance failure, besides a bug-triggering input, 
the user usually provides another input similar to the bad one, 
but with a much better performance.
Therefore, statistical debugging is a natural fit for performance problems. 
After evaluating widely used predicates and 
two representative statistical models, 
we found that branch predicate plus the two statistical models can effectively 
pinpoint wrong branch decisions and inefficient loops, 
causing the diagnosed performance failures. 
We also found that
due to the unique nature of inefficient loops, 
sampling can lower runtime overhead without sacrificing diagnosis latency, 
which is very different from functional failure diagnosis.
We then built LDoctor~\cite{Song17ICSE} to provide more 
fine-grained diagnosis information and fixing strategies for inefficient loops. 
LDoctor takes suspicious loops reported by 
statistical debugging as inputs, 
applies hybridized static and dynamic analysis to 
balance accuracy and performance, 
and relies on sampling and other designed optimizations 
to further reduce runtime overhead. 
Evaluation results under real-world inefficient loops 
show that LDoctor covers most common root causes, 
reports few false positives, and incurs a low runtime overhead. 

\vspace{-.1in}
\paragraph{Performance Bug Fixing.}
Intel Xeon Phi coprocessors have recently been introduced 
as new members of the manycore family.
Compared with GPUs, Xeon Phi coprocessors are easier to program,
because they provide x86 compatibility and support many different programming models.
Adding simple pragmas can help developers offload existing parallel loops 
to Xeon Phi coprocessors. 
However, this does not result in better performance, 
and too many performance bugs are contained in 
offloaded parallel loops. 
After careful investigation, we designed three source-to-source code transformations to 
hide data transfer overhead between CPUs and coprocessors, 
to regularize loops with irregular memory accesses, 
and to efficiently transfer large pointer-based data structures, respectively~\cite{Song14MICRO}.
Our transformations can automatically fix performance bugs contained in offloaded parallel loops. 
Experimental results show that the designed transformations can benefit 9 
out of 12 evaluated benchmarks and improve performance by 1.16x - 52.21x. 
This work won MICRO-47 (2014) best paper runner-up.


\vspace{-.1in}
\section{Current and Near-Future Research Directions (Post-PhD)}
After joining the Pennsylvania State University as an assistant professor,
I have expanded my research directions to investigate
emerging reliability and security issues.  
In the following, I will focus on two main research themes 
to discuss my initial progress and my plans for the near future. 

\vspace{-.1in}
\subsection{Fighting Issues in New Programming Languages}
Many new programming languages have been created since 2000, 
with unique language features trying to accomplish their desired design goals.
With increasing adoption, it is important to understand 
whether the design goals are really achieved 
and whether there are unique bugs in these languages. 
Working together with my collaborator (Yiying Zhang from Purdue University), 
I mainly look into Go and Rust, because 
these two languages are popular choices 
of developers when implementing 
data center applications and safety-critical systems. 
 
\vspace{-.1in}
\paragraph{Understanding Concurrency Issues in Go}
Go is a statically-typed new programming language that aims to provide 
a simple, efficient, and safe way to build multi-threaded software.
For this purpose, Go centers its multi-threading design 
around two principles: 1) making threads (called goroutines) lightweight 
and easy to create and 2) using explicit messaging (called channel) 
to communicate across threads. With these design principles, 
Go proposes not only new primitives and new libraries 
but also new implementation of existing semantics.
Due to its powerful and expressive concurrency mechanisms, 
Go is widely used in data center environments,
like building container systems and implementing microservices.

It is important to understand how Go's new concurrency mechanisms impact concurrency bugs, 
since concurrency bugs are difficult to debug and severely degrade 
the reliability of concurrent systems. 
We conducted the first empirical study on Go concurrency bugs, based on 171 bugs collected from  
six famous Go software 
including Docker, Kubernetes and gRPC~\cite{go-asplos}.
More than half of the studied bugs are caused by non-traditional, Go-specific problems. 
Apart from root causes of these bugs, we also studied their fixes, 
performed experiments to reproduce them, 
and evaluated them with two publicly-available Go bug detectors.
Overall, our study provides a better understanding on Go's concurrency models
with many valuable findings and implications, such as message passing is more likely to 
cause blocking bugs and new programming models
that Go introduced to ease multi-threaded programming can themselves 
become the reasons of more concurrency bugs. 
Our study can guide future researchers and practitioners in writing better, 
more reliable Go software and in developing novel debugging tools for Go.

\vspace{-.1in}
\paragraph{Understanding Safety Issues in Rust}
Rust is a young programming language designed for system software development. 
It aims to provide safety guarantees like high-level languages 
and performance efficiency
like low-level languages. The core design of Rust is a set
of strict safety rules enforced by compile-time checking. To
support more low-level controls, Rust allows programmers
to bypass these compiler checks to write unsafe code. 
Programmers can also encapsulate their unsafe code within safe
function interfaces (i.e., interior unsafe).

It is important to understand how well Rust’s safety (and
unsafe) mechanisms work in real programs. We performed
the first empirical study of Rust by close, manual inspection of 583 unsafe code usages 
and 103 bugs in five open-source Rust projects, five widely-used Rust libraries, and two
online security databases~\cite{sosp-boqin,yu2019fearless}. Our study answers three important 
questions: how and why do programmers write unsafe
code, what memory-safety issues real Rust programs have,
and what concurrency bugs Rust programmers make. 
We find that programmmers often choose to write unsafe code for good performance 
and ease of programming, although all memory-safety bugs involve unsafe code, 
most of them also involve safe code, and 
most concurrency bugs are caused by programmers' misunderstanding of Rust's safety features 
and compiler rules. 
For all the above three aspects, we make insightful suggestions 
to future Rust programmers and language developers. 
One key take-away is that a theoretically proven safe language may be hard to 
use in reality, and as a result, the language includes less safe features, which eventually cause
problems in the real world.

\vspace{-.1in}
\paragraph{Future Plan (1-3 years)}
Understanding real-world issues in Go and Rust is just a starting point.
My next step is to build tools to automatically pinpoint previously unknown issues. 
From the developers' perspective, patching an issue is the final 
stage to really improve systems' realibility and safety. 
I plan to explore the following key research questions for Go and Rust 
under the guidance of our empirical studies: 
1) How to statically detect concurrency bugs, 
while solving or avoiding traditional, difficult problems, 
like alias analysis in a large software
and identifying variables accessed by multiple threads?
2) How to dynamically detect concurrency bugs with a low 
runtime overhead to deploy the detection in production runs?
3) How to extend existing techniques designed for traditional programming languages, 
such as fuzz testing and symbolic execution?
4) How to synthesize patches for identified bugs and validate 
their correctness and performance?

\vspace{-.1in}
\subsection{Applying Data Analytics to Big Malware Data}
VirusTotal is a free online malware scanning service that applies 
more than 60 state-of-the-art anti-virus engines to analyze user-submitted files (or URLs)
and sends aggragated detection results (e.g., labels, malware families) back to users. 
VirusTotal receives millions of submissions every day and 
provides public access to all its submitted files and analysis results.
VirusTotal is a valuable resource to study and understand real-world malware 
and anti-virus engines, especially in today's big data era~\cite{Song16ApSys}.  

VirusTotal is heavily used by the research community for malware sample collection,
data labelling and system evaluation. 
Our literature review~\cite{vt-jianjun} shows that 
VirusTotal has already been used by more than 100 academic 
papers in top-tier venues 
in security, network and measurements including S\&P, CCS, 
USENIX Security, NDSS, IMC, and WWW.
A closer examination shows that researchers use VirusTotal under two assumptions. 
First, VirusTotal periodically updates its engines, 
so that VirusTotal labels possibly change over time due to the updates. 
However, most researchers simply query VirusTotal once, 
with the assumption that label changes are rare, 
generating negligible impact. 
Second, the more than 60 VirusTotal engines 
may disagree with each other, 
such as some engines label a submission as benign, 
while others label the submission as malicious. 
To aggregate labels from different engines and determine whether a submission is malicious,
researchers usually use a threshold --- 
if at least $t$ engines return a malicious label,
researchers regard the submission as malicious. 
When computing $t$, most researchers assume all engines have the 
same labeling capability and treat all engines equally. 

I am working with my collobrator (Gang Wang from UIUC) to look into 
whether the two assumptions are true in reality
and whether researchers are using VirusTotal in the right way~\cite{vt-jianjun,imc-peng}. 
After comprehensive measurements on two large data sets collected from VirusTotal
over a long period of time, 
we find that label changes are very common on VirusTotal and 
some engines' labeling decisions are highly influenced by others. 
Our findings are contrary to the common practices when researchers are using VirusTotal
and raise an alarm for future VirusTotal users. 
Besides measurements, we also use machine learning models to quantify the influence 
among vendors and to predict when a submission's labels will not change any more, 
the goal to judge whether a submission's labels have become stable and can be trusted. 
%Our models can help researchers and practioners better utilize the malware labeling 
%functionality of VirusTotal. 

\vspace{-.1in}
\paragraph{Future Plan (1-3 years)}
In the near future, I will continue my work to help researchers and practioners 
better interpret and utilize VirusTotal data. 
For example, I will examine whether using a threshold to aggregate VirusTotal labels   
is reasonable and integrate vendors' influence into the label aggregation. 
As another example, an engine may have different capabilities when analyzing 
different types of malware. I plan to build a fine-grained 
engine influence model after categorizing malware based on their families. 
I will also try to defend possible attacks launched through VirusTotal. 
More and more engines are built based on machine learning models these days. 
VirusTotal serves as a valuable resource for training data.
Since VirusTotal is publicly accessible to everyone, it also provides 
a convenient channel for attackers to distribute adversial samples, 
which can enable real malware to bypass machine learning models inside VirusTotal engines.
I plan to investigate how to detect adversial samples on VirusTotal.  

\begin{table*}[th]
\centering
\begin{tabular}{|c|c|c|c|}
\hline
						\multicolumn{4}{|c|}{\textbf{Other Completed Works}}  \\ \hline
\multicolumn{1}{|c|}{Automated Bug Fixing}   & \multicolumn{1}{c|}{Mining Software Repositories} & \multicolumn{1}{c|}{Software Profiling} 		& \multicolumn{1}{c|}{Software Debloating} \\ \hline
 				~\cite{jin11afix}	   	    &              ~\cite{Gu15FSE}                                     &         ~\cite{icse-boqin}                              &  ~\cite{bangwen}~\cite{salad}        \\ \hline
\end{tabular}
\end{table*}



\section{Long-term Research Directions}

While my ongoing projects and recent research plans have been discussed as above,
in the long term, I plan to explore how to apply machine learning to solve
traditional problems in system and programming languages area and 
how to address performance, reliability and security issues 
in emerging computing platforms. 


\vspace{-.1in}
\paragraph{Machine Learning for Systems and Programming Languages.} 
In system and programming languages areas, 
many traditional problems (e.g., alias analysis, cache eviction, I/O scheduling) 
are tentatively solved by heuristic rules, which have to be revisted as 
workloads change over time. 
Systems themselves can generate a huge amount of data, 
and these data can automatically be labeled by 
referring system status in the near future. 
Therefore, I plan to explore how to leverage machine learning
to address traditional problems in system area.
I plan to work on the following fundermental questions:
(1) How to select suitable representations for system status 
and system decisions to feed traditional system problems 
into machine learning models?
(2) How to select machine learning models suitable 
to the system problems to be solved?
(3) How to explain trained machine learning models 
and verify their underlying properties?
(4) How to efficiently integrate machine 
learning solutions to existing systems?
My industrial collobrations (e.g., ByteDance, FireEye) 
can help access systems in a large scale and test my machine learning 
solutions under production-run workloads. 


\vspace{-.1in}
\paragraph{Improve Performance, Reliability and Security for Emerging Computing Platforms.} 

Novel computing platforms appear every year. 
They leverage new hardware and new programming models,
providing different services for people's daily life. 
Of course, there could be completely new challenges 
in performance, reliability and security.
For example, there are billions of connected IoT devices in use worldwide, 
and the number is still growing rapidly. 
The real device hardware may not be available during development, 
causing troubles during in-house tesing. 
The computation power on these devices is usually limited, 
so that many existing security tools cannot be deployed on these devices. 
Many of them are home devices and wearable devices,  
so that privacy is a bigger concerns compared with traditional platforms (e.g. personal computer, 
cloud cluster).
I plat to identify and solve reliability, performance and security issues 
for emerging platforms. 
I will understand the problems firstl. Then I will systematically evaluate existing techniques.
Eventually, I will design novel solutions. 
My technical background and expertise can help achieve this research agenda. 



\newpage
\bibliographystyle{plainyr-rev}
\bibliography{rs}
\end{document}
